% Options for packages loaded elsewhere
\PassOptionsToPackage{unicode}{hyperref}
\PassOptionsToPackage{hyphens}{url}
%
\documentclass[
]{article}
\usepackage{amsmath,amssymb}
\usepackage{iftex}
\ifPDFTeX
  \usepackage[T1]{fontenc}
  \usepackage[utf8]{inputenc}
  \usepackage{textcomp} % provide euro and other symbols
\else % if luatex or xetex
  \usepackage{unicode-math} % this also loads fontspec
  \defaultfontfeatures{Scale=MatchLowercase}
  \defaultfontfeatures[\rmfamily]{Ligatures=TeX,Scale=1}
\fi
\usepackage{lmodern}
\ifPDFTeX\else
  % xetex/luatex font selection
\fi
% Use upquote if available, for straight quotes in verbatim environments
\IfFileExists{upquote.sty}{\usepackage{upquote}}{}
\IfFileExists{microtype.sty}{% use microtype if available
  \usepackage[]{microtype}
  \UseMicrotypeSet[protrusion]{basicmath} % disable protrusion for tt fonts
}{}
\makeatletter
\@ifundefined{KOMAClassName}{% if non-KOMA class
  \IfFileExists{parskip.sty}{%
    \usepackage{parskip}
  }{% else
    \setlength{\parindent}{0pt}
    \setlength{\parskip}{6pt plus 2pt minus 1pt}}
}{% if KOMA class
  \KOMAoptions{parskip=half}}
\makeatother
\usepackage{xcolor}
\usepackage[margin=1in]{geometry}
\usepackage{color}
\usepackage{fancyvrb}
\newcommand{\VerbBar}{|}
\newcommand{\VERB}{\Verb[commandchars=\\\{\}]}
\DefineVerbatimEnvironment{Highlighting}{Verbatim}{commandchars=\\\{\}}
% Add ',fontsize=\small' for more characters per line
\usepackage{framed}
\definecolor{shadecolor}{RGB}{248,248,248}
\newenvironment{Shaded}{\begin{snugshade}}{\end{snugshade}}
\newcommand{\AlertTok}[1]{\textcolor[rgb]{0.94,0.16,0.16}{#1}}
\newcommand{\AnnotationTok}[1]{\textcolor[rgb]{0.56,0.35,0.01}{\textbf{\textit{#1}}}}
\newcommand{\AttributeTok}[1]{\textcolor[rgb]{0.13,0.29,0.53}{#1}}
\newcommand{\BaseNTok}[1]{\textcolor[rgb]{0.00,0.00,0.81}{#1}}
\newcommand{\BuiltInTok}[1]{#1}
\newcommand{\CharTok}[1]{\textcolor[rgb]{0.31,0.60,0.02}{#1}}
\newcommand{\CommentTok}[1]{\textcolor[rgb]{0.56,0.35,0.01}{\textit{#1}}}
\newcommand{\CommentVarTok}[1]{\textcolor[rgb]{0.56,0.35,0.01}{\textbf{\textit{#1}}}}
\newcommand{\ConstantTok}[1]{\textcolor[rgb]{0.56,0.35,0.01}{#1}}
\newcommand{\ControlFlowTok}[1]{\textcolor[rgb]{0.13,0.29,0.53}{\textbf{#1}}}
\newcommand{\DataTypeTok}[1]{\textcolor[rgb]{0.13,0.29,0.53}{#1}}
\newcommand{\DecValTok}[1]{\textcolor[rgb]{0.00,0.00,0.81}{#1}}
\newcommand{\DocumentationTok}[1]{\textcolor[rgb]{0.56,0.35,0.01}{\textbf{\textit{#1}}}}
\newcommand{\ErrorTok}[1]{\textcolor[rgb]{0.64,0.00,0.00}{\textbf{#1}}}
\newcommand{\ExtensionTok}[1]{#1}
\newcommand{\FloatTok}[1]{\textcolor[rgb]{0.00,0.00,0.81}{#1}}
\newcommand{\FunctionTok}[1]{\textcolor[rgb]{0.13,0.29,0.53}{\textbf{#1}}}
\newcommand{\ImportTok}[1]{#1}
\newcommand{\InformationTok}[1]{\textcolor[rgb]{0.56,0.35,0.01}{\textbf{\textit{#1}}}}
\newcommand{\KeywordTok}[1]{\textcolor[rgb]{0.13,0.29,0.53}{\textbf{#1}}}
\newcommand{\NormalTok}[1]{#1}
\newcommand{\OperatorTok}[1]{\textcolor[rgb]{0.81,0.36,0.00}{\textbf{#1}}}
\newcommand{\OtherTok}[1]{\textcolor[rgb]{0.56,0.35,0.01}{#1}}
\newcommand{\PreprocessorTok}[1]{\textcolor[rgb]{0.56,0.35,0.01}{\textit{#1}}}
\newcommand{\RegionMarkerTok}[1]{#1}
\newcommand{\SpecialCharTok}[1]{\textcolor[rgb]{0.81,0.36,0.00}{\textbf{#1}}}
\newcommand{\SpecialStringTok}[1]{\textcolor[rgb]{0.31,0.60,0.02}{#1}}
\newcommand{\StringTok}[1]{\textcolor[rgb]{0.31,0.60,0.02}{#1}}
\newcommand{\VariableTok}[1]{\textcolor[rgb]{0.00,0.00,0.00}{#1}}
\newcommand{\VerbatimStringTok}[1]{\textcolor[rgb]{0.31,0.60,0.02}{#1}}
\newcommand{\WarningTok}[1]{\textcolor[rgb]{0.56,0.35,0.01}{\textbf{\textit{#1}}}}
\usepackage{graphicx}
\makeatletter
\def\maxwidth{\ifdim\Gin@nat@width>\linewidth\linewidth\else\Gin@nat@width\fi}
\def\maxheight{\ifdim\Gin@nat@height>\textheight\textheight\else\Gin@nat@height\fi}
\makeatother
% Scale images if necessary, so that they will not overflow the page
% margins by default, and it is still possible to overwrite the defaults
% using explicit options in \includegraphics[width, height, ...]{}
\setkeys{Gin}{width=\maxwidth,height=\maxheight,keepaspectratio}
% Set default figure placement to htbp
\makeatletter
\def\fps@figure{htbp}
\makeatother
\setlength{\emergencystretch}{3em} % prevent overfull lines
\providecommand{\tightlist}{%
  \setlength{\itemsep}{0pt}\setlength{\parskip}{0pt}}
\setcounter{secnumdepth}{-\maxdimen} % remove section numbering
\ifLuaTeX
  \usepackage{selnolig}  % disable illegal ligatures
\fi
\usepackage{bookmark}
\IfFileExists{xurl.sty}{\usepackage{xurl}}{} % add URL line breaks if available
\urlstyle{same}
\hypersetup{
  pdftitle={Exercise5},
  pdfauthor={Haoran Xu},
  hidelinks,
  pdfcreator={LaTeX via pandoc}}

\title{Exercise5}
\author{Haoran Xu}
\date{2025-02-05}

\begin{document}
\maketitle

\begin{Shaded}
\begin{Highlighting}[]
\FunctionTok{library}\NormalTok{(dplyr) }\CommentTok{\# A Grammar of Data Manipulation}
\FunctionTok{library}\NormalTok{(evd) }\CommentTok{\# Functions for Extreme Value Distributions }
\FunctionTok{library}\NormalTok{(ggplot2) }\CommentTok{\# Create Elegant Data Visualisations Using the Grammar of Graphics}
\end{Highlighting}
\end{Shaded}

\subsection{1. What do we mean when we say that the logit probability
has a closed
form?}\label{what-do-we-mean-when-we-say-that-the-logit-probability-has-a-closed-form}

Closed form is a term to describe the solutions to the integral that can
be precisely derived using integration. The analytical solution to the
logit model is the exact value of the integral, which can be written as:
\[
F(x;\mu, \sigma) = \frac{1}{1 + e^{- (x - \mu) / \sigma}}
\]

The Binomial Logit model can be written as: \[
P_i = P(\epsilon_n < V_i - V_j) = \frac{1}{1 + e^{- (V_i - V_j)}} = \frac{e^{V_i}}{e^{V_i} + e^{V_j}}
\] \(P_i\) above is the logit probability of choosing alternative \(i\),

The multinomial Logit models can be written as: \[
P_i = \frac{e^{V_i}}{\sum_{j}^{J}e^{V_j}}
\]

Both of them are closed forms.

\subsection{2. Why is it that we can set the dispersion parameter in the
logit probabilities to
one?}\label{why-is-it-that-we-can-set-the-dispersion-parameter-in-the-logit-probabilities-to-one}

Because the dispersion parameter in probability distribution will not
alter the ordinality of choosing different alternatives. Since it can
only affect the cardinality, we can arbitrarily set it as 1 for
convenience.

\subsection{3. Suppose that a choice set consists of two alternatives,
travel by car (c) and
travel}\label{suppose-that-a-choice-set-consists-of-two-alternatives-travel-by-car-c-and-travel}

by blue bus (bb). The utilities of these two modes are the same, that is
\[
V_{c} = V_{bb}
\] What are the probabilities of choosing these two modes?

If the \emph{reference} function is applied, then probabilities of
choosing these two modes will be the both \(0.5\), which is the same, as
\mu has been absorbed into one of the utility functions.

If the \emph{reference} function is not applied, the value of
\mu becomes important in deciding the difference of \(V_{c}\) and
\(V_{bb}\). If \mu is 0, meaning that the probability distribution of
the difference of random utility (i.e., \(\epsilon_{n}\)) is centered at
0, the probabilities of the two modes would be equally \(0.5\). If
\mu is not 0, both of their probabilities will not be \(0.5\).

\subsection{4. Suppose that the transit operator of the blue buses in
Question 3 decides to
introduce}\label{suppose-that-the-transit-operator-of-the-blue-buses-in-question-3-decides-to-introduce}

a new service, namely a red bus. This red bus is identical to the blue
bus in every respect except the color. Under these new conditions, what
are the logit probabilities of choosing these modes?

\subsection{5. Discuss the results of introducing a new mode in the
choice process
above.}\label{discuss-the-results-of-introducing-a-new-mode-in-the-choice-process-above.}

\end{document}
